\documentclass[UTF8]{ctexart}
\usepackage{amsmath, amssymb, amsthm}

\newtheorem{theorem}{定理}
\newtheorem{lemma}{引理}
\newtheorem{corollary}{推论}
\newtheorem{definition}{定义}
\newtheorem{example}{例}

\title{拉格朗日插值}
\author{黄知远}
\date{\today}

\begin{document}

\maketitle

\tableofcontents

\section{拉格朗日插值的代数形式}

\begin{lemma}
    对于 n 次多项式 f(x), 只需要 n + 1 个不同函数值就能确定 f(x) 的各项系数.
\end{lemma}
\begin{proof}
    设 \( n \) 次多项式  
    \[
    P(x) = a_0 + a_1x + a_2x^2 + \dots + a_nx^n
    \]

    若已知其在 \( n+1 \) 个互不相同点上的取值  
    \[
    P(x_i) = y_i \quad (i = 0,1,\dots,n)
    \]

    则有线性方程组  
    \[
    \begin{bmatrix}
    1 & x_0 & x_0^2 & \cdots & x_0^n \\
    1 & x_1 & x_1^2 & \cdots & x_1^n \\
    \vdots & \vdots & \vdots & \ddots & \vdots \\
    1 & x_n & x_n^2 & \cdots & x_n^n
    \end{bmatrix}
    \begin{bmatrix}
    a_0 \\ a_1 \\ \vdots \\ a_n
    \end{bmatrix}
    =
    \begin{bmatrix}
    y_0 \\ y_1 \\ \vdots \\ y_n
    \end{bmatrix}
    \]

    左矩阵为 Vandermonde 矩阵,其行列式  
    \[
    \prod_{0 \le i < j \le n} (x_j - x_i) \ne 0
    \]
    当 \( x_i \) 互异时可逆,方程组唯一可解。

    故 \( n \) 次多项式由 \( n+1 \) 个点唯一确定。
\end{proof}

对于 \( n \) 次多项式 \( P(x) = a_0 + a_1x + a_2x^2 + \dots + a_nx^n \),
如果已知\(P(x_0),P(x_2),\dots,P(x_n)\)的值, 我们可以构造:
\[
    P(x) = \sum_{0}^{n} P(x_i) \prod_{j \neq i}^{}\frac{x-x_j}{x_i-x_j}
\]
带入 \((x_i, P(x_i))\) 验证, 发现这个形式是正确的, 因此这是对\(P(x)\)一种合法表示.

\section{单点拉格朗日插值}

\subsection{已知值取值连续时的做法}

若已知量 \(x_0 \dots\ x_n\) 连续, 且按照大小顺序排序, 我们可以将第一节提到的合法表示改写成下面的形式:
\[
    P(x) = \sum_{0}^{n} (-1)^{n-i} y_i \frac{pre_{i-1} \cdot suf_{i+1}}{fac_i \cdot fac_{n-i}}
\]
\[
    pre_i = \prod_{0}^{i} x - x_i
\]
\[
    suc_i = \prod_{i}^{n} x - x_i
\]
\[
    fac_i = \prod_{1}^{i} j
\]

通过前后缀优化,我们可以 \(O(n\log n)\) 计算这个式子,其中 \(\log n\) 来源于计算逆元,当然你可以
用 \(O(1)\) 逆元的技巧将其优化到 \(O(n)\) .

\subsection{已知值不连续时的做法}
笔者只知道朴素的 \(O(n^2)\) 做法, 不做赘述.

\end{document}
