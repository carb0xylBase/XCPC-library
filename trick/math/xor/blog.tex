\documentclass[UTF8]{ctexart}
\usepackage{amsmath, amssymb, amsthm}

\newtheorem{theorem}{定理}
\newtheorem{lemma}{引理}
\newtheorem{corollary}{推论}
\newtheorem{definition}{定义}
\newtheorem{example}{例}
\newtheorem*{conclusion}{结论}

\title{关于 xor 的 trick}
\author{黄知远, chatgpt}
\date{\today}

\begin{document}

\maketitle

\tableofcontents

\section{\([a, b] \oplus k\) 的结果形成 \(O(\log n)\) 个区间}

这个结论的证明以及代码实现需要用到 dyadic 分解.

\subsection{dyadic 概念及性质}

区间 \([a, b]\) 的 dyadic 分解是将其唯一地划分为若干个形如 \([x, x + 2^t - 1]\) 的子区间, 
使得每个子区间的左端点 \(x\) 满足 \(2^t \mid x\).

对于任意非负整数 \(t\) , 我们将区间 \([k2^t, (k+1)2^t-1]\) 定义为 \(t\) 的一个对齐块.
定义函数 \(F_p(k)\) , 表示将 \(k\) 对应的对齐块中的元素与 \(p\) 进行异或, 得到的数的集合.

\begin{lemma}
    \(F_p(k)\) 一定是一个关于 \(t\) 的对齐块.
\end{lemma}
\begin{proof}
    每一个对齐块一定包含前 \(t\) 个二进制位的全排列, 因此在异或上某个数后, 前 \(t\) 个二进制位会被
    映射成一个新的全排列, 而其余二进制位异或出的结果显然是相同的, 所以异或的结果任然是一个对齐块.
\end{proof}

\begin{lemma}
    dyadic 分解产生的区间有 \(O(\log_2 (b - a + 1))\) 个.
\end{lemma}
\begin{proof}
    笔者不是数学专业的, 就写个感性证明了. 假设某一次删除了大小为 \(2^t\) 的块, 那么下一次是可以
    选择大小为 \(2^{t+1}\) 的块的, 因此选择的区间大小会不断翻倍, 那么大概只能选 \(\log n\) 次.
    但是有时 \(b\) 的大小不允许我们选更大的块, 也就是说选择的块的大小开始下降了, 这一部分的操作
    次数和上升阶段是类似的, 所以最多也是只有 \(\log n\) 次. 那么总的次数就是 \(O(\log n)\) 次.
\end{proof}

\begin{conclusion}
    由上述分析可知,区间 $[a,b]$ 可被至多 $O(\log(b-a+1))$ 个 dyadic 块覆盖, 每个块都被异或操作
    映射到一个连续区间, 所以总共映射为 \(O(\log (b - a + 1))\) 个区间.
\end{conclusion}


\subsection{dyadic 分解的算法实现}

假设当前有一个区间 \([a, b]\). 当 \(a \le b\) 时, 执行以下步骤:
\begin{enumerate}
    \item 找到最小的整数 \(t\),使得 \(2^t \mid a\) 且 \(a + 2^t - 1 \le b\);
    \item 取出子区间 \([a,\, a + 2^t - 1]\);
    \item 将 \(a\) 更新为 \(a + 2^t\), 继续上述过程, 直到 \(a > b\).
\end{enumerate}

\(O((\log n) ^ 2)\) 的做法是显然的.

\section{\((kp+1)\oplus(p-1)\) 的单调性}

\begin{conclusion}
    对任意质数 $p$, 当 $k$ 取同一奇偶性时, $(kp+1)\oplus(p-1)$ 随 $k$ 严格单调增加.
\end{conclusion}

\begin{proof}
    令 $a=p-1$, $b_k=kp+1$, $A_k=b_k\mathbin{\&}a$. 利用恒等式 $x\oplus y = x+y-2(x\mathbin{\&}y)$, 得
    \[
    f(k):=b_k\oplus a=(kp+1)+(p-1)-2A_k=(k+1)p-2A_k.
    \]
    若 $k_2-k_1=2t$ 且 $t\ge1$, 则
    \[
    f(k_2)-f(k_1)=(k_2-k_1)p-2(A_{k_2}-A_{k_1})=2tp-2\Delta A,
    \]
    其中 $0\le A_k\le p-1$, 故 $\Delta A\le p-1$. 因此
    \[
    f(k_2)-f(k_1)\ge 2p-2(p-1)=2>0,
    \]
    从而在相同奇偶性的 $k$ 上, $f(k)$ 严格递增. 对 $p=2$ 的情形也成立.
\end{proof}

\end{document}