\documentclass[UTF8]{ctexart}
\usepackage{amsmath, amssymb, amsthm}

\newtheorem{theorem}{定理}
\newtheorem{lemma}{引理}
\newtheorem{corollary}{推论}
\newtheorem{definition}{定义}
\newtheorem{example}{例}
\newtheorem*{conclusion}{结论}

\title{关于凸函数的 trick}
\author{黄知远, chatgpt}
\date{\today}

\begin{document}

\maketitle

\tableofcontents

\section{序列的 min-plus 卷积}

给定两个序列 \( a = (a_0, a_1, \dots, a_{n-1}) \)、\( b = (b_0, b_1, \dots, b_{m-1}) \),  
它们的 min-plus 卷积定义为
\[
c_i = \min_{\substack{0 \le j \le n-1 \\ 0 \le k \le m-1 \\ j + k = i}} (a_j + b_k),
\quad (0 \le i \le n + m - 2).
\]

通常情况下, 只有 \(O(nm)\) 的暴力做法.

\subsection{a 和 b 均为凸序列时}

根据凸序列的定义, 凸序列的差分显然是单调增的, 如果 a 和 b 都是凸序列,
那么考虑其差分, 我们计算 \(c_i\) 时必然考虑前 i 个最小的差分,
于是我们只需要做一个双指针进行贪心的选择就可以了, 时间复杂度 \(O(n + m)\) .

更通俗的说, 结论是 c 的差分数组, 就是 a 和 b 的差分数组合并在一起, 然后从
小到大排序的结果. 于是我们发现一个事, 两个凸序列的 min-plus 卷积仍然是一个
凸序列.

\end{document}